\documentclass{scrartcl}
%Aus dem LaTex Template der Universit�t Stuttgart
%------------------------------------------------
\usepackage[utf8]{inputenc}
\usepackage[T1]{fontenc}
\usepackage{cmap}
\usepackage[ngerman]{babel}
\usepackage{graphicx}
\usepackage[pdftex,hyperref,dvipsnames]{xcolor}
\usepackage{listings}
\usepackage[a4paper,lmargin={2cm},rmargin={2cm},tmargin={3.5cm},bmargin = {2.5cm},headheight = {4cm}]{geometry}
\usepackage{amsmath,amssymb,amstext,amsthm}
\usepackage[lined,algonl,boxed]{algorithm2e}
\usepackage{tikz}
\usepackage{hyperref}
\usepackage{url}
\usepackage[inline]{enumitem} % Erm�glicht �ndern der enum Item Zahlen
\usepackage[headsepline]{scrpage2} 
\usepackage{algorithmic} % F�r Pseudocode
\usepackage{ marvosym } % f�r Pfeil(e)
\usepackage{booktabs} % F�r die sch�neren Booktabs-Tabellen
\usepackage{tikz}
\usepackage{pdfpages}
\usepackage{blindtext}
\usepackage{scrextend}
\usepackage{natbib} % Yannis hat das importiert; TODO: nachfragen, zu was das gut ist
\pagestyle{scrheadings} 
\usetikzlibrary{automata,positioning}

\begin{document}
	% Counter für das Blatt und die Aufgabennummer.
% Ersetze die Nummer des Übungsblattes und die Nummer der Aufgabe
% den Anforderungen entsprechend.
% Beachte:
% \setcounter{countername}{number}: Legt den Wert des Counters fest
% \stepcounter{countername}: Erhöht den Wert des Counters um 1.
\newcounter{sheetnr}
\setcounter{sheetnr}{6} % Nummer des Übungsblattes
\newcounter{exnum}
\setcounter{exnum}{1} % Nummer der Aufgabe

% Befehl für die Aufgabentitel
\newcommand{\exercise}[1]{\section*{Aufgabe \theexnum\stepcounter{exnum} #1}} % Befehl für Aufgabentitel

% Formatierung der Kopfzeile
% \ohead: Setzt rechten Teil der Kopfzeile mit
% Namen und Matrikelnummern aller Bearbeiter
\ohead{Yannis Blosch (3256958)\\
Lukas Heiland (3269754)\\
Nils Boike (3257520)}
% \chead{} kann mittleren Kopfzeilen Teil sezten
% \ihead: Setzt linken Teil der Kopfzeile mit
% Modulnamen, Semester und Übungsblattnummer
\ihead{Systemkonzepte und -programmierung\\
Wintersemester 2017/18\\
Blatt \thesheetnr}

	
	\section*{Aufgabe 2}
		\paragraph*{a)}
			\begin{itemize}
				\item $P_1$ wird von $P_2$ in eine Warteschlange eingereiht und $P_2$ geht in den kritischen Abschnitt, wenn $P_2$ fertig ist sendet er $"reply"$ an $P_1$
				\item $P_2$ sendet $"reply"$ an $P_1$ und wartet auf Antwort von $P_1$
				\item $P_2$ sendet $"reply"$ an $P_1$
			\end{itemize}
	
		\paragraph*{b)}
			$P_4 \rightarrow P_2 \rightarrow P_3 \rightarrow P_6 \rightarrow P_1 \rightarrow P_5$
	
		\paragraph*{c)}
			Nehmen wir an, P1 (Ticketnummer 5) und P2 (Ticketnummer 10) möchten gleichzeitig in den kritischen Abschnitt eintreten, die beiden Prozesse senden jeweils eine request nachricht, P1 erhält nun aufgrund seiner kleineren Ticketnummer den reply von P2. P1 tritt also in den kritischen Abschnitt ein und sendet danach einen Reply an P2. Danach setzt P1 aber nun seine neue Ticketnummer auf 8 und gelangt so wieder in den Kritischen Abschnitt zusammen mit P2. Siehe Vorlesung Nachrichtenaustausch S. 55-56.
	
		\paragraph*{d)}
			!!!!!!!!!!!!!!!!!!!!!!!!!!!!!!
	
		\paragraph*{e)}
		Prozesse der Anzahl n, senden an jeweils n-1 prozesse Nachrichten x mal\\
		$(n \cdot (n-1)) \cdot x \Rightarrow n^2$\pagebreak
	
	
	\section*{Aufgabe 3}
		\paragraph*{a)}
		Wenn zum Zeitpunkt der Tokenverteilung mehrere Requests von Prozessen vorliegen, kann ein Prozess aushungern, da der Token zufällig zugeteilt wird.\\
		Bsp.: P1 und P2 requesten im Wechsel immer wieder, sodass P3 nie den Token bekommt (s.u.):\\[2em]
	
		\begin{figure}[h]
			\includegraphics[scale=0.7]{aufgabe3a.pdf}
		\end{figure}
	
		\paragraph*{b)}
			\subparagraph*{i.}
				Es müssen P7 und P8 atomar ausgeführt werden in der main Methode genauso wie q3 und q4 in der recieve Methode.
	
			\subparagraph*{ii.}
				Wird erst p7 und danach q3 und q4 ausgeführt ist der wechselseitige Ausschluss nicht mehr gewährleistet. 
			\subparagraph*{iii.}
	
\end{document}
